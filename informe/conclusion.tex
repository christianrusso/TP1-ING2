\section{Conclusi'on}

% opinión retrospectiva sobre las herramientas
El desarrollo del sistema nos facilitó la apreciación de la utilidad de varias de las herramientas
que usamos como los gráficos y las heurísticas de diseño. Sin embargo, en varios momentos
el orden en el que realizamos las tareas consideramos que nos perjudicó. Por ejemplo, el
subdividir el trabajo en stories sin haber pensado en el diseño del sistema antes, produjo
una correspondencia pobre entre las stories que habíamos definido y las tareas que finalmente
tuvimos que realizar.

% desafíos
Los gráficos resultaron ser una buena base para discutir las opciones de diseño de forma barata
sin tener que invertir el tiempo que hubiera tomado programar cada una de las alternativas. Sin embargo,
sucedió varias veces que no nos percatábamos de ciertos problemas hasta que no los implementábamos.
Además, uno de los mayores desafíos que encontramos en el TP fue pensar diseños que satisfacieran
todas las heurísticas de diseño que teníamos presentes (alta cohesión, bajo acoplamiento, alta modularidad, etc.).
La mayoría de las veces tuvimos que evaluar las ventajas y desventajas de cada alternativa para finalmente 
decidir la que nos pareciera mejor para la circunstancia particular y los posibles ejes de cambio.

% target process
Finalmente, nos gustaría comentar que la utilización de la herramienta de Scrum Target Process no nos 
pareció que nos ayudara a organizarnos o mejorar. Probablemente esto se deba a que el proyecto fue demasiado
corto para poder utilizar las herramientas de seguimiento de Target Process durante las retrospectivas
y organizarnos mejor.
